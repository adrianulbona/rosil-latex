% \chapter{Project's Objectives and Specification}
\chapter{Obiective și specificații}
\label{cap:obiective-specificatii}

Acest capitol conține descrierea detaliată a temei de cercetare propriu-zise, formulată exact, cu obiective clare și specificații, pe 2-3 pagini și eventuale figuri explicative. Titlul nu e neapărat impus și, de asemenea, capitolul poate fi inclus ca subcapitol în Capitolul~\ref{cap:Introducere}, dacă se potrivește.

Reprezintă cca. 5--10\% din lucrare.

% \section{Objectives}
\section{Obiective}

Obiectivele proiectului sunt lucrurile care se dorește a fi realizate, ca urmare a abordării temei lucrării de disertație. În general numărul de obiective este proporțional cu timpul de care dispunem. Exemple generice:
\begin{enumerate}
  \item Analiza critică a soluțiilor existente pentru problema abordată și identificarea posibile limitări ale acestora.
  \item Propunerea unor soluții la (o parte) din problemele identificate. 
  \item Implementarea unui/unor prototipuri de validare și testare a soluțiilor propuse 
  \item Identificarea unor teme de dezvoltare și cercetare ulterioare
  \item \dots
\end{enumerate}


% \section{Project Specification}
\section{Specificații}

% \subsection{Functional Specification}
\subsection{Specificații funcționale}

Soluția noastră:
\begin{itemize}
  \item va face următoarele ...
  \item va oferi următoarea funcționalitate \dots
  \item va fi bazată pe modelul \dots (client-server) 
  \item va fi implementată în C, Java etc.
  \item \dots
\end{itemize}


% \subsection{Non-Functional Specification}
\subsection{Specificații non-funcționale}

Soluția/Prototipul dezvoltat ar trebuie, de asemenea, să aibă următoarele caracteristici non-funcționale (exemple):
\begin{itemize}
  \item să aibă următoarea performanță
  \item să fie ușor/intuitiv de utilizat
  \item să fie adoptată pe scară largă 
  \item \dots
\end{itemize}




