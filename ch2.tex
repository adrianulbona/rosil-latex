\chapter{Obiective}
\label{cap:obiective-specificatii}
\section{Obiective}
Contribuția principală asteptată de la acest proiect ține de dezvoltarea unei metode prin care se pot obține silabisiri ale cuvintelor scrise, cu precizie ridicată. 

Pe lângă acest obiectiv general trebuie avute în vedere și următoarele cerințe: 

\begin{itemize}
\item \textit{Genericitate la nivel de limba}. Se urmărește o identificarea unei soluții generale, ca să funcționeze cel puțin la nivelul familiilor de limbi.
\item \textit{Scalabilitate la nivelul timpului de execuție}. Este necesar ca despărțirile în silabe să se realizeze cât mai rapid posibil, mai ales în contextul sintezei de voce. 
\item \textit{Înglobarea excepțiilor lingvistice în cadrul soluției}. Acest obiectiv, deși antitetic cu primul, trebuie cel puțin avut în vedere.
\item \textit{Tratarea silabificărilor dependente de context}. Cel puțin în cadrul limbii române, despărțirea în silabe poate fi ambiguă, depinzând de contextul în care este realizată.
\item \textit{Identificarea unui set de metrici pentru validarea rezultatelor.}

\end{itemize}
