\chapter{Introducere}
\label{cap:Introducere}
În cadrul acestui proiect se încearcă identificarea unei metode care rezolvă problema silabificării. 

O astfel de metodă este de interes într-o serie de domenii conexe lingvisticii, spre exemplu în cadrul editoarelor de text sau pentru sinteza artificială a limbajului (audio).

Soluția propusă se folosește de metode de învățare supervizată bazându-se tipare secventiale. 

Ce se scrie aici:
\begin{itemize}
    \item Contextul
    \item Conturarea/Descrierea domeniului exact al temei
    \item Se răspunde la întrebările: \textbf{ce} (s-a făcut)?, \textbf{de ce} (s-a făcut, adică motivația; ce se rezolvă, la ce este util, etc.)?, \textbf{cum} (s-a făcut, adică particularitățile abordării, prezentate sumar).
    \item Introducerea se termină cu o descriere a conținutului lucrării, de genul: Cap X descrie ..., Cap Y prezintă ...
    \item Introducerea reprezintă o sinteză a lucrării, din care cititorul trebuie să-și poată da bine seama dacă lucrarea prezintă sau nu interes pentru el. 
    \item Se poate organiza pe subsecțiuni, dacă se dorește, după exemplul de mai jos, dar nu e obligatoriu asta, având în vedere dimensiunea mică
    \item reprezintă cca 5\% din lucrare (nu mai mult de 2-4 pagini)
\end{itemize}

\section{Context}

Despre contextul în care este abordată și se aplică tema lucrării.

% \section{Motivation}
\section{Motivație}
De ce este utilă abordarea temei? Ce probleme rezolvă și ce rezultate poate aduce?

% \section{Report's Structure}
\section{Structura lucrării}
Capitolul~\ref{cap:obiective-specificatii} prezintă obiectivele \dots. Capitolul~\ref{cap:studiu-bibliografic} descrie \dots. În capitolul~\ref{cap:fund-teoretice} sunt prezentate \dots.
