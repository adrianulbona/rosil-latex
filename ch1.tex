\chapter{Introducere}
\label{cap:Introducere}

Silabisirea ca problemă prezintă un interes ridicat, soluțiile având aplicabilitate în cadrul unei multitudini de domenii. 

În cadrul acestei lucrări va fi analizată acestă problemă și posibilele soluții.

\section{Context}

O metodă eficientă de silabisirea poate avea aplicabilitate în diverse domenii, printre acestea poate fi menționată sinteza artificială de voce sau despărțirile cu aplicabilitate în tehnoredactare. Aplicabilitatea directă a soluției propusă în cadrul acestei lucrări ține de sinteza artificială de voce.

O metodă de silabisire precisă este o componentă necesară în vederea sintezei artificială de voce. Acestă cerință ține de natura limbilor vorbite, în care, cuvintele nu sunt pronunțate literă cu literă, ci mai degraba ca secvențe de silabe, silaba putând fi asociată cu elementul atomic al pronunției.

Pornind de la această observație, o analiză a secvențialității silabelor din cadrul cuvintelor unei limbi, poate fi utilizată în vederea identificării de tipare frecvente, iar pe baza acestora să fie construite soluții pentru problema silabisirii.  

Astfel, acestă problemă poate fi plasată în contextul extragerii de cunoștinte, după cum va fi prezentat în capitolele următoare.

\section{Motivație}

Dezvoltarea unei metode eficiente si precise de silabisire poate oferi informații valoroase specialiștilor din domeniul lingvisticii și prelucrării limbajului natural, dar și utilizarea unei astfel de metode în cadrul unor aplicații practice oferă o motivație suficient de puternică pentru a investiga această problemă. 

Dincolo de faptul că metoda este aplicată direct pentru limba română, se dorește un nivel suficient de ridicat de generalitate, astfel încât să poată fi aplicată și in contextul altor limbi.

\section{Structura lucrării}

În cadrul capitolului al 2-lea va fi prezentață specificația de la care s-a pornit în cadrul acestui proiect iar în capitolul al 3-lea se vor aduce în revistă studii similare și alte referinte aflate în legătură metoda propusă. În cadrul capitolului al 4-lea vor fi prezentate o serie de concepte de bază utilizate pentru a descrie metoda propusă iar în capitolul al 5-lea va fi prezentată în detaliu metoda propusă. În cadrul capitolului al 6-lea vor fi prezentate o serie de detalii de implementare. Rezultatele experimentale sunt prezentate în cadrul capitolului al 7-lea iar în ultimei părți vor fi prezentate o serie de concluzii, cât si câteva idei pentru dezvoltări ulterioare. 