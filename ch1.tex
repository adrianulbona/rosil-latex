\chapter{Introducere}
\label{cap:Introducere}


În cadrul acestei lucrări de dizertație se prezintă o metodă inovativă prin care se pot obține silabisiri ale cuvintelor, cu grad ridicat de acuratețe. 

\section{Context}

O metodă eficientă de silabisirea poate fi poate avea aplicabilitate în diverse domenii, printre acestea poate fi menționată sinteza artificială de voce sau despărțirile cu aplicabilitate în tehnoredactare. Aplicabilitate imediată a metodei descrise se află în domeniul sintezei de voce.

\section{Motivație}

Dezvoltarea unei metode eficiente de silabisire poate oferi informații prețioase specia-liștilor din domeniul lingvisticii dar și utilizarea unei astfel de metode în cadrul unor aplicații practice oferă o motivație suficient de puternică pentru a investiga această problemă. 

Dincolo de faptul că metoda este aplicată direct pentru limba română, se dorește un nivel suficient de ridicat de generalitate, astfel încât să poată fi aplicată și in contextul altor limbi.

\section{Structura lucrării}

În cadrul capitolului 2 va fi presentață specificația de la care s-a pornit în cadrul acestui proiect iar în capitolul 3 se vor aduce în revistă studii similare și alte referinte aflate în legătură metoda propusă. În cadrul capitolului 4 vor fi prezentate o serie de concepte de bază utilizate pentru a descrie metoda propusă iar în capitolul 5 va fi prezentată în detaliu metoda propusă cât si o serie de detalii de implementare. Rezultatele experimentale sunt prezentate în cadrul capitolului 4 iar în ultimul capitor vor fi prezentate o serie de concluzii, cât si câteva idei pentru dezvoltări ulterioare. 