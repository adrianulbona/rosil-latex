\chapter{Prezentarea contribuțiilor autorului}
\label{cap:contributii}

\section{Precizări asupra conținutului și a modului de organizare}

Titlul acestui capitol nu este unul impus și nici nu corespunde neapărat unui singur capitol. Titlul indică mai degrabă o parte (importantă și centrală, de altfel) a lucrării, în care se prezintă ceea ce s-a realizat efectiv: contribuțiile autorului. Organizarea acestei părți este dependentă și specifică fiecărei lucrări în parte și este stabilită de către fiecare autor după cum i se pare mai potrivit pentru tema lui. Ea poate cuprinde prezentarea unor concepte teoretice (unelte sau tehnici matematice folosite în lucrare, prezentarea sau introducerea unor concepte teoretice etc.), o analiză a diferitelor metode/algoritmi/tehnologii etc. luate în considerare sau dezvoltate de către autor, o prezentare a unui design (mai mult sau mai puțin detaliat) sau chiar detalii a unei eventuale implementări/prototip, dacă e cazul.

Trebuie remarcat însă faptul că această parte reprezintă contribuția personală a autorului, chiar dacă ea constă de exemplu doar dintr-o analiză comparativă a unor metode/algoritmi, și în nici un caz ea nu poate fi sinteza unor texte preluate din alte surse. Prin urmare, orice informații sunt prezentate aici, ele trebuie să corespundă cel puțin unei interpretări/analize critice personale a autorului, dacă nu chiar unor idei originale ale acestuia. 

\subsection{Dimensiune}

Împreună cu capitolul (partea) următor reprezintă cca. 70\% din lucrare. 


\section{Examples: lists, figures, tables, equations}

Așa arată o listă de elemente nenumerotate:
\begin{itemize}
  \item element 1
  \item element 2
  \item \dots
\end{itemize}


Așa arată o listă de elemente numerotare:
\begin{itemize}
  \item element 1
  \item element 2
  \item \dots
\end{itemize}


Așa arată o listă în text: 
\begin{inparaenum}[(\itshape 1 \upshape)]
  \item element 1, 
  \item element 2, 
  \item \dots
\end{inparaenum}

\textbf{Atenție}: orice tabel, figura sau ecuație (formulă) trebuie referite \textit{explicit} în text explicit (de genul: în Figura X este ulustrat \dots, în Tabelul Y se poate vedea \dots), pentru că Latex le poate plasa chiar și pe altă pagină decât acolo unde vrem noi să ne referim la ele. Vedeți exemple de mai jos!

Tabelul~\ref{table:example} ilustrează un exemplu de tabel. Un editor on-line de tabele poate fi găsit la \url{http://www.tablesgenerator.com/}. 

\begin{table}[t]
\centering                          % tabel centrat 
\begin{tabular}{|c|c|c|c|}          % 4 coloane centrate 
\hline\hline                        % linie orizontala dubla
Case & Method\#1 & Method\#2 & Method\#3 \\ [0.5ex]   % inserare tabel
%heading
\hline                              % linie orizontal simpla
1 & 50 & 837 & 970 \\               % corpul tabelului 
2 & 47 & 877 & 230 \\
3 & 31 & 25 & 415 \\[1ex]           % [1ex] adds vertical space
\hline                              
\end{tabular}
\caption{Nonlinear Model Results}   % titlul tabelului
\label{table:example}                % \label{table:nonlin} introduce eticheta folosita pentru referirea tabelului in text; referirea in text se va face cu \ref{table:nonlin}
\end{table}

În Figura~\ref{fig:exemplu} 


Formula~(\ref{eq:example}) arată modul de calcul al lui $\Delta$:
\begin{equation} \label{eq:example}
    \Delta =\sum_{i=1}^N w_i (x_i - \bar{x})^2 .
\end{equation}


Algoritmul~\ref{alg:example} este un exemplu de descriere pseudo-cod a unui algoritm, preluat de la \href{http://en.wikibooks.org/wiki/LaTeX/Algorithms#Typesetting_using_the_algorithm2e_package}{http://en.wikibooks.org/wiki/LaTeX}. El utilizează pachetul \textit{algorithm2e}. Alternativ, puteți utiliza pachetele \textit{algorithmic} sau \textit{program}. 

\begin{algorithm}
 \KwData{this text}
 \KwResult{how to write algorithm with \LaTeX2e }
 initialization\;
 \While{not at end of this document}{
  read current\;
  \eIf{understand}{
   go to next section\;
   current section becomes this one\;
   }{
   go back to the beginning of current section\;
  }
 }
 \caption{How to write algorithms}
 \label{alg:example}
\end{algorithm}
