\chapter{Conluzii}
\label{cap:concluzii}
Cuprinde:

\begin{itemize}
 \item un rezumat al contribuțiilor aduse: ce s-a realizat, relativ la ce s-a propus, în ce constă experiența acumulată, care au fost punctele dificile atinse și rezolvată, recomandări pentru alții care abordează tema, la ce este bun ce s-a obținut etc.
 
 \item a analiză critică a rezultatelor obținute: avantaje, dezavantaje, limitări
 
 \item o descriere a posibilelor dezvoltări și îmbunătățiri ulterioare
\end{itemize}

Poate fi organizat pe secțiuni, dacă se dorește.

Se întinde pe aproximativ 1-2 pagini. 









