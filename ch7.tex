\chapter{Posibile dezvoltări ulterioare}

În cadrul acestei lucrări au fost realizate o serie de experimente prin care s-a demonstrat că tiparele secvențiale frecvente pot fi utilizate pentru a prezice silabisiri ale cuvintelor. La nivel de implementare, prototipul realizat a fost modelat cu intenția de putea fi extins ulterior. În cele ce urmează vor fi schițate o serie de idei care ar putea fi explorate pentru a creste calitatea și performanța metodei.


\section{Dezvoltări la nivelul metodei}

\subsection{Utilizarea suportului tiparelor frecvente}
Dintre cele trei strategii de predicție prezentate anterior, niciuna dintre ele nu utilizează suportul tiparelor identificate la nivelul grafurilor de tipare. Cu sigurantă ar trebui explorată posibitatea ca această informație să fie utilizată pentru a discrimina între lanțurile închise de tipare.

\subsection{Utilizarea de tipare cu goluri}
Tiparele identificate în prototipul curent sunt continue și conțin silabe întregi. Ar putea fi explorată posibilitatea ca tiparele să conțină doar părți de silabe. Intuitiv, astfel de tipare ar trebui sa fie mai generale. 
Pe de altă parte, astfel de tipare ar schimba felul în care este realizată potrivirea tiparelor în cadrul cuvintelor.

\subsection{Sibabisire sensibilă la context}
După cum a mai fost menționat în cateva rânduri, problema silabisirii prezintă un nivel ridicat de ambiguitate. Pentru a trece peste acest impediment, o posibilă soluție este introducerea unor contexte în cadrul cărora să nu mai existe ambiguitate. 

Concret silabisirea fonetică și cea ortografică ar putea să definească fiecare câte un context de silabisire. La nivelul punerii în practică a acestei idei, este suficientă o grupare a cuvintelor silabisite în funcție de context, iar ulterior, când se doreste o predicție de silabisire să se specifice și contextul în care se dorește a fi realizată predicția.

\subsection{Validarea genericității} 

O posibilă dezvoltare ulterioară ține de evaluarea genericității metodei. În cadrul experimentelor cuprinse în cadrul acestei lucrări au fost analizată precizia pentru limba română și engleză. Ar fi de interes o evaluare în cazul unor limbi non-indo-europene.

\subsection{Evaluarea impactului caracterelor speciale} 

În cazul unei limbi cum este limba română, nu a fost analizată influența diacriticelor asupra metodei. Impactul acestora asupra tiparelor frecvente și ulterior predicțiilor ar putea fi măsurat cu usurință. 

\subsection{Dezvoltarea unei metode de compresie pe baza tiparelor frecvente}

Odată identificate tiparele frecvente din cadrul unei limbi, acestea ar putea fi folosite alături de metoda de potrivire de tipare din cadrul acestei lucrări pentru a dezvolta o metodă de compresie utilizând aceste tipare indexate. 

\section{Dezvoltări la nivelul implementării}

\subsection{Implementarea unei aplicații WEB}

Anterior a fost prezentat felul în care poate fi construit un serviciu WEB care să realizeze predicții de silabisire. Ar fi de interes extinderea acestor servicii, prin care să fie expus si graful de tipare frecvente identificat pentru un anumit cuvânt. Odată accesibil, graful de tipare ar putea fi vizualizat în cadrul unei aplicații WEB. Aplicația web ar putea fi utilizată de specialiști din domeniul lingvisticii, dar ar ușura și trasarea întregii metode.

\subsection{Paralelizarea intregii aplicații}

La nivelul implementării curente, au fost utilizate o serie de procesări paralele, utilizând mecanismele specifice Java 8 (parallel streams, lamba functions, concurrent collections), acestea ar putea fi utilizate la nivelul întregii aplicații. 


\chapter{Concluzii}
\label{cap:concluzii}

În cadrul acestei lucrări a fost explorată problema silabisirii și a fost propusă o metodă prin care această problemă poate fi rezolvată. Au fost descrise conceptele teoretice utilizate la construcția soluției, a fost realizată o implementare a metodei și au fost evaluate rezultatele acesteia. 

Trebuie menționat că soluția propusă, deși nu oferă cea mai ridicată precizie, comparativ cu alte abordări, vine cu o serie de avantaje: genericitate (nedepinzând de regulile specifice ale unei limbi), performanță la nivelul timpului de execuție și extensibilitate.

Dincolo de problema silabisirii, genericitatea soluției poate fi dusă un pas mai departe și s-ar putea încerca aplicarea ei și în cadrul altor probleme pentru care, pornind de la un set de tipare secvențiale frecvente, se dorește recompunerea unor secvențe cu ajutorul acestor tipare.