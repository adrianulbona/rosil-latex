\chapter{Concluzii si direcții posibile de urmat}
\label{cap:concluzii}

În cadrul acestei lucrări a fost explorată problema silabisirii și a fost propusă o metodă prin care această problemă poate fi rezolvată. Au fost descrise conceptele teoretice utilizate la construcția soluției, a fost realizată o implementare a metodei și au fost evaluate rezultatele acesteia. 

Trebuie menționat că soluția propusă, deși nu oferă cea mai ridicată precizie, comparativ cu alte abordări, vine cu o serie de avantaje: genericitate, nedepinzând de regulile specifice ale unei limbi, performanță la nivelul timpului de execuție și extensibilitate.

Pentru viitor munca existentă ar putea fi continuată în câteva direcții:

\begin{itemize}
\item Definirea unor \textit{contexte} de silabisire. Silabisirea nu se bazează pe un set de reguli clare, dar asociind anumite reguli cu un anumit context și aborband acele contexte diferit, s-ar putea îmbunătăți precizia soluției.

\item Validarea metodei în cadrul unor limbi non-indoeuropene. 

\item Dezvoltarea unei aplicații web prin care grafurile de tipare sa poată fi analizate de către specialiștii din lingvistică. 



\end{itemize}









